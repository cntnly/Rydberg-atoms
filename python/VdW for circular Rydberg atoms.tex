
% Default to the notebook output style

    


% Inherit from the specified cell style.




    
\documentclass{article}

    
    
    \usepackage{graphicx} % Used to insert images
    \usepackage{adjustbox} % Used to constrain images to a maximum size 
    \usepackage{color} % Allow colors to be defined
    \usepackage{enumerate} % Needed for markdown enumerations to work
    \usepackage{geometry} % Used to adjust the document margins
    \usepackage{amsmath} % Equations
    \usepackage{amssymb} % Equations
    \usepackage[mathletters]{ucs} % Extended unicode (utf-8) support
    \usepackage[utf8x]{inputenc} % Allow utf-8 characters in the tex document
    \usepackage{fancyvrb} % verbatim replacement that allows latex
    \usepackage{grffile} % extends the file name processing of package graphics 
                         % to support a larger range 
    % The hyperref package gives us a pdf with properly built
    % internal navigation ('pdf bookmarks' for the table of contents,
    % internal cross-reference links, web links for URLs, etc.)
    \usepackage{hyperref}
    \usepackage{longtable} % longtable support required by pandoc >1.10
    \usepackage{booktabs}  % table support for pandoc > 1.12.2
    

    
    
    \definecolor{orange}{cmyk}{0,0.4,0.8,0.2}
    \definecolor{darkorange}{rgb}{.71,0.21,0.01}
    \definecolor{darkgreen}{rgb}{.12,.54,.11}
    \definecolor{myteal}{rgb}{.26, .44, .56}
    \definecolor{gray}{gray}{0.45}
    \definecolor{lightgray}{gray}{.95}
    \definecolor{mediumgray}{gray}{.8}
    \definecolor{inputbackground}{rgb}{.95, .95, .85}
    \definecolor{outputbackground}{rgb}{.95, .95, .95}
    \definecolor{traceback}{rgb}{1, .95, .95}
    % ansi colors
    \definecolor{red}{rgb}{.6,0,0}
    \definecolor{green}{rgb}{0,.65,0}
    \definecolor{brown}{rgb}{0.6,0.6,0}
    \definecolor{blue}{rgb}{0,.145,.698}
    \definecolor{purple}{rgb}{.698,.145,.698}
    \definecolor{cyan}{rgb}{0,.698,.698}
    \definecolor{lightgray}{gray}{0.5}
    
    % bright ansi colors
    \definecolor{darkgray}{gray}{0.25}
    \definecolor{lightred}{rgb}{1.0,0.39,0.28}
    \definecolor{lightgreen}{rgb}{0.48,0.99,0.0}
    \definecolor{lightblue}{rgb}{0.53,0.81,0.92}
    \definecolor{lightpurple}{rgb}{0.87,0.63,0.87}
    \definecolor{lightcyan}{rgb}{0.5,1.0,0.83}
    
    % commands and environments needed by pandoc snippets
    % extracted from the output of `pandoc -s`
    \DefineVerbatimEnvironment{Highlighting}{Verbatim}{commandchars=\\\{\}}
    % Add ',fontsize=\small' for more characters per line
    \newenvironment{Shaded}{}{}
    \newcommand{\KeywordTok}[1]{\textcolor[rgb]{0.00,0.44,0.13}{\textbf{{#1}}}}
    \newcommand{\DataTypeTok}[1]{\textcolor[rgb]{0.56,0.13,0.00}{{#1}}}
    \newcommand{\DecValTok}[1]{\textcolor[rgb]{0.25,0.63,0.44}{{#1}}}
    \newcommand{\BaseNTok}[1]{\textcolor[rgb]{0.25,0.63,0.44}{{#1}}}
    \newcommand{\FloatTok}[1]{\textcolor[rgb]{0.25,0.63,0.44}{{#1}}}
    \newcommand{\CharTok}[1]{\textcolor[rgb]{0.25,0.44,0.63}{{#1}}}
    \newcommand{\StringTok}[1]{\textcolor[rgb]{0.25,0.44,0.63}{{#1}}}
    \newcommand{\CommentTok}[1]{\textcolor[rgb]{0.38,0.63,0.69}{\textit{{#1}}}}
    \newcommand{\OtherTok}[1]{\textcolor[rgb]{0.00,0.44,0.13}{{#1}}}
    \newcommand{\AlertTok}[1]{\textcolor[rgb]{1.00,0.00,0.00}{\textbf{{#1}}}}
    \newcommand{\FunctionTok}[1]{\textcolor[rgb]{0.02,0.16,0.49}{{#1}}}
    \newcommand{\RegionMarkerTok}[1]{{#1}}
    \newcommand{\ErrorTok}[1]{\textcolor[rgb]{1.00,0.00,0.00}{\textbf{{#1}}}}
    \newcommand{\NormalTok}[1]{{#1}}
    
    % Define a nice break command that doesn't care if a line doesn't already
    % exist.
    \def\br{\hspace*{\fill} \\* }
    % Math Jax compatability definitions
    \def\gt{>}
    \def\lt{<}
    % Document parameters
    \title{VdW for circular Rydberg atoms}
    
    
    

    % Pygments definitions
    
\makeatletter
\def\PY@reset{\let\PY@it=\relax \let\PY@bf=\relax%
    \let\PY@ul=\relax \let\PY@tc=\relax%
    \let\PY@bc=\relax \let\PY@ff=\relax}
\def\PY@tok#1{\csname PY@tok@#1\endcsname}
\def\PY@toks#1+{\ifx\relax#1\empty\else%
    \PY@tok{#1}\expandafter\PY@toks\fi}
\def\PY@do#1{\PY@bc{\PY@tc{\PY@ul{%
    \PY@it{\PY@bf{\PY@ff{#1}}}}}}}
\def\PY#1#2{\PY@reset\PY@toks#1+\relax+\PY@do{#2}}

\expandafter\def\csname PY@tok@ni\endcsname{\let\PY@bf=\textbf\def\PY@tc##1{\textcolor[rgb]{0.60,0.60,0.60}{##1}}}
\expandafter\def\csname PY@tok@il\endcsname{\def\PY@tc##1{\textcolor[rgb]{0.40,0.40,0.40}{##1}}}
\expandafter\def\csname PY@tok@sb\endcsname{\def\PY@tc##1{\textcolor[rgb]{0.73,0.13,0.13}{##1}}}
\expandafter\def\csname PY@tok@s2\endcsname{\def\PY@tc##1{\textcolor[rgb]{0.73,0.13,0.13}{##1}}}
\expandafter\def\csname PY@tok@k\endcsname{\let\PY@bf=\textbf\def\PY@tc##1{\textcolor[rgb]{0.00,0.50,0.00}{##1}}}
\expandafter\def\csname PY@tok@mh\endcsname{\def\PY@tc##1{\textcolor[rgb]{0.40,0.40,0.40}{##1}}}
\expandafter\def\csname PY@tok@nc\endcsname{\let\PY@bf=\textbf\def\PY@tc##1{\textcolor[rgb]{0.00,0.00,1.00}{##1}}}
\expandafter\def\csname PY@tok@gt\endcsname{\def\PY@tc##1{\textcolor[rgb]{0.00,0.27,0.87}{##1}}}
\expandafter\def\csname PY@tok@cm\endcsname{\let\PY@it=\textit\def\PY@tc##1{\textcolor[rgb]{0.25,0.50,0.50}{##1}}}
\expandafter\def\csname PY@tok@se\endcsname{\let\PY@bf=\textbf\def\PY@tc##1{\textcolor[rgb]{0.73,0.40,0.13}{##1}}}
\expandafter\def\csname PY@tok@s1\endcsname{\def\PY@tc##1{\textcolor[rgb]{0.73,0.13,0.13}{##1}}}
\expandafter\def\csname PY@tok@kt\endcsname{\def\PY@tc##1{\textcolor[rgb]{0.69,0.00,0.25}{##1}}}
\expandafter\def\csname PY@tok@vg\endcsname{\def\PY@tc##1{\textcolor[rgb]{0.10,0.09,0.49}{##1}}}
\expandafter\def\csname PY@tok@gp\endcsname{\let\PY@bf=\textbf\def\PY@tc##1{\textcolor[rgb]{0.00,0.00,0.50}{##1}}}
\expandafter\def\csname PY@tok@nt\endcsname{\let\PY@bf=\textbf\def\PY@tc##1{\textcolor[rgb]{0.00,0.50,0.00}{##1}}}
\expandafter\def\csname PY@tok@cp\endcsname{\def\PY@tc##1{\textcolor[rgb]{0.74,0.48,0.00}{##1}}}
\expandafter\def\csname PY@tok@c\endcsname{\let\PY@it=\textit\def\PY@tc##1{\textcolor[rgb]{0.25,0.50,0.50}{##1}}}
\expandafter\def\csname PY@tok@sc\endcsname{\def\PY@tc##1{\textcolor[rgb]{0.73,0.13,0.13}{##1}}}
\expandafter\def\csname PY@tok@s\endcsname{\def\PY@tc##1{\textcolor[rgb]{0.73,0.13,0.13}{##1}}}
\expandafter\def\csname PY@tok@nl\endcsname{\def\PY@tc##1{\textcolor[rgb]{0.63,0.63,0.00}{##1}}}
\expandafter\def\csname PY@tok@kr\endcsname{\let\PY@bf=\textbf\def\PY@tc##1{\textcolor[rgb]{0.00,0.50,0.00}{##1}}}
\expandafter\def\csname PY@tok@ne\endcsname{\let\PY@bf=\textbf\def\PY@tc##1{\textcolor[rgb]{0.82,0.25,0.23}{##1}}}
\expandafter\def\csname PY@tok@o\endcsname{\def\PY@tc##1{\textcolor[rgb]{0.40,0.40,0.40}{##1}}}
\expandafter\def\csname PY@tok@bp\endcsname{\def\PY@tc##1{\textcolor[rgb]{0.00,0.50,0.00}{##1}}}
\expandafter\def\csname PY@tok@ss\endcsname{\def\PY@tc##1{\textcolor[rgb]{0.10,0.09,0.49}{##1}}}
\expandafter\def\csname PY@tok@gi\endcsname{\def\PY@tc##1{\textcolor[rgb]{0.00,0.63,0.00}{##1}}}
\expandafter\def\csname PY@tok@gd\endcsname{\def\PY@tc##1{\textcolor[rgb]{0.63,0.00,0.00}{##1}}}
\expandafter\def\csname PY@tok@nv\endcsname{\def\PY@tc##1{\textcolor[rgb]{0.10,0.09,0.49}{##1}}}
\expandafter\def\csname PY@tok@gs\endcsname{\let\PY@bf=\textbf}
\expandafter\def\csname PY@tok@kn\endcsname{\let\PY@bf=\textbf\def\PY@tc##1{\textcolor[rgb]{0.00,0.50,0.00}{##1}}}
\expandafter\def\csname PY@tok@sd\endcsname{\let\PY@it=\textit\def\PY@tc##1{\textcolor[rgb]{0.73,0.13,0.13}{##1}}}
\expandafter\def\csname PY@tok@na\endcsname{\def\PY@tc##1{\textcolor[rgb]{0.49,0.56,0.16}{##1}}}
\expandafter\def\csname PY@tok@nd\endcsname{\def\PY@tc##1{\textcolor[rgb]{0.67,0.13,1.00}{##1}}}
\expandafter\def\csname PY@tok@si\endcsname{\let\PY@bf=\textbf\def\PY@tc##1{\textcolor[rgb]{0.73,0.40,0.53}{##1}}}
\expandafter\def\csname PY@tok@kc\endcsname{\let\PY@bf=\textbf\def\PY@tc##1{\textcolor[rgb]{0.00,0.50,0.00}{##1}}}
\expandafter\def\csname PY@tok@gr\endcsname{\def\PY@tc##1{\textcolor[rgb]{1.00,0.00,0.00}{##1}}}
\expandafter\def\csname PY@tok@sh\endcsname{\def\PY@tc##1{\textcolor[rgb]{0.73,0.13,0.13}{##1}}}
\expandafter\def\csname PY@tok@mi\endcsname{\def\PY@tc##1{\textcolor[rgb]{0.40,0.40,0.40}{##1}}}
\expandafter\def\csname PY@tok@kp\endcsname{\def\PY@tc##1{\textcolor[rgb]{0.00,0.50,0.00}{##1}}}
\expandafter\def\csname PY@tok@c1\endcsname{\let\PY@it=\textit\def\PY@tc##1{\textcolor[rgb]{0.25,0.50,0.50}{##1}}}
\expandafter\def\csname PY@tok@vi\endcsname{\def\PY@tc##1{\textcolor[rgb]{0.10,0.09,0.49}{##1}}}
\expandafter\def\csname PY@tok@nn\endcsname{\let\PY@bf=\textbf\def\PY@tc##1{\textcolor[rgb]{0.00,0.00,1.00}{##1}}}
\expandafter\def\csname PY@tok@no\endcsname{\def\PY@tc##1{\textcolor[rgb]{0.53,0.00,0.00}{##1}}}
\expandafter\def\csname PY@tok@vc\endcsname{\def\PY@tc##1{\textcolor[rgb]{0.10,0.09,0.49}{##1}}}
\expandafter\def\csname PY@tok@nf\endcsname{\def\PY@tc##1{\textcolor[rgb]{0.00,0.00,1.00}{##1}}}
\expandafter\def\csname PY@tok@kd\endcsname{\let\PY@bf=\textbf\def\PY@tc##1{\textcolor[rgb]{0.00,0.50,0.00}{##1}}}
\expandafter\def\csname PY@tok@w\endcsname{\def\PY@tc##1{\textcolor[rgb]{0.73,0.73,0.73}{##1}}}
\expandafter\def\csname PY@tok@nb\endcsname{\def\PY@tc##1{\textcolor[rgb]{0.00,0.50,0.00}{##1}}}
\expandafter\def\csname PY@tok@go\endcsname{\def\PY@tc##1{\textcolor[rgb]{0.53,0.53,0.53}{##1}}}
\expandafter\def\csname PY@tok@mf\endcsname{\def\PY@tc##1{\textcolor[rgb]{0.40,0.40,0.40}{##1}}}
\expandafter\def\csname PY@tok@ge\endcsname{\let\PY@it=\textit}
\expandafter\def\csname PY@tok@gh\endcsname{\let\PY@bf=\textbf\def\PY@tc##1{\textcolor[rgb]{0.00,0.00,0.50}{##1}}}
\expandafter\def\csname PY@tok@mo\endcsname{\def\PY@tc##1{\textcolor[rgb]{0.40,0.40,0.40}{##1}}}
\expandafter\def\csname PY@tok@err\endcsname{\def\PY@bc##1{\setlength{\fboxsep}{0pt}\fcolorbox[rgb]{1.00,0.00,0.00}{1,1,1}{\strut ##1}}}
\expandafter\def\csname PY@tok@cs\endcsname{\let\PY@it=\textit\def\PY@tc##1{\textcolor[rgb]{0.25,0.50,0.50}{##1}}}
\expandafter\def\csname PY@tok@sx\endcsname{\def\PY@tc##1{\textcolor[rgb]{0.00,0.50,0.00}{##1}}}
\expandafter\def\csname PY@tok@ow\endcsname{\let\PY@bf=\textbf\def\PY@tc##1{\textcolor[rgb]{0.67,0.13,1.00}{##1}}}
\expandafter\def\csname PY@tok@gu\endcsname{\let\PY@bf=\textbf\def\PY@tc##1{\textcolor[rgb]{0.50,0.00,0.50}{##1}}}
\expandafter\def\csname PY@tok@sr\endcsname{\def\PY@tc##1{\textcolor[rgb]{0.73,0.40,0.53}{##1}}}
\expandafter\def\csname PY@tok@m\endcsname{\def\PY@tc##1{\textcolor[rgb]{0.40,0.40,0.40}{##1}}}

\def\PYZbs{\char`\\}
\def\PYZus{\char`\_}
\def\PYZob{\char`\{}
\def\PYZcb{\char`\}}
\def\PYZca{\char`\^}
\def\PYZam{\char`\&}
\def\PYZlt{\char`\<}
\def\PYZgt{\char`\>}
\def\PYZsh{\char`\#}
\def\PYZpc{\char`\%}
\def\PYZdl{\char`\$}
\def\PYZhy{\char`\-}
\def\PYZsq{\char`\'}
\def\PYZdq{\char`\"}
\def\PYZti{\char`\~}
% for compatibility with earlier versions
\def\PYZat{@}
\def\PYZlb{[}
\def\PYZrb{]}
\makeatother


    % Exact colors from NB
    \definecolor{incolor}{rgb}{0.0, 0.0, 0.5}
    \definecolor{outcolor}{rgb}{0.545, 0.0, 0.0}



    
    % Prevent overflowing lines due to hard-to-break entities
    \sloppy 
    % Setup hyperref package
    \hypersetup{
      breaklinks=true,  % so long urls are correctly broken across lines
      colorlinks=true,
      urlcolor=blue,
      linkcolor=darkorange,
      citecolor=darkgreen,
      }
    % Slightly bigger margins than the latex defaults
    
    \geometry{verbose,tmargin=1in,bmargin=1in,lmargin=1in,rmargin=1in}
    
    

    \begin{document}
    
    
    \maketitle
    
    

    
    \begin{Verbatim}[commandchars=\\\{\}]
{\color{incolor}In [{\color{incolor}15}]:} \PY{o}{\PYZpc{}}\PY{k}{pylab} \PY{n}{inline}
         \PY{c}{\PYZsh{}path = \PYZsq{}/home/2kome/Desktop/testdeck/test1/python/\PYZsq{}}
         \PY{n}{path} \PY{o}{=} \PY{l+s}{\PYZsq{}}\PY{l+s}{C:}\PY{l+s+se}{\PYZbs{}\PYZbs{}}\PY{l+s}{Users}\PY{l+s+se}{\PYZbs{}\PYZbs{}}\PY{l+s}{r14}\PY{l+s+se}{\PYZbs{}\PYZbs{}}\PY{l+s}{Documents}\PY{l+s+se}{\PYZbs{}\PYZbs{}}\PY{l+s}{GitHub}\PY{l+s+se}{\PYZbs{}\PYZbs{}}\PY{l+s}{test}\PY{l+s+se}{\PYZbs{}\PYZbs{}}\PY{l+s}{python}\PY{l+s}{\PYZsq{}} \PY{c}{\PYZsh{}should be change to the propriate folder}
         \PY{k}{if} \PY{n}{path} \PY{o+ow}{not} \PY{o+ow}{in} \PY{n}{sys}\PY{o}{.}\PY{n}{path}\PY{p}{:}
             \PY{n}{sys}\PY{o}{.}\PY{n}{path}\PY{o}{.}\PY{n}{append}\PY{p}{(}\PY{n}{path}\PY{p}{)}
         \PY{k+kn}{import} \PY{n+nn}{os}
         \PY{n}{os}\PY{o}{.}\PY{n}{chdir}\PY{p}{(}\PY{n}{path}\PY{p}{)}
         \PY{k+kn}{from} \PY{n+nn}{imp} \PY{k+kn}{import} \PY{n+nb}{reload}
         \PY{c}{\PYZsh{}\PYZpc{}config InlineBackend.figure\PYZus{}format=\PYZsq{}png\PYZsq{}}
         \PY{c}{\PYZsh{}pylab.rcParams[\PYZsq{}figure.figsize\PYZsq{}] = (10.0, 8.0)}
         \PY{n}{pylab}\PY{o}{.}\PY{n}{rcParams}\PY{p}{[}\PY{l+s}{\PYZsq{}}\PY{l+s}{savefig.dpi}\PY{l+s}{\PYZsq{}}\PY{p}{]} \PY{o}{=} \PY{l+m+mi}{200}
\end{Verbatim}

    \begin{Verbatim}[commandchars=\\\{\}]
Populating the interactive namespace from numpy and matplotlib
    \end{Verbatim}

    \begin{Verbatim}[commandchars=\\\{\}]
WARNING: pylab import has clobbered these variables: ['random', 'e', 'linalg', 'fft', 'power', 'info']
`\%matplotlib` prevents importing * from pylab and numpy
    \end{Verbatim}

    \section{\texorpdfstring{\(|60C60C\rangle\)}{\textbar{}60C60C\textbackslash{}rangle}}\label{c60crangle}

Define atom pair to calculate \(C6\). Here is \(|60C60C\rangle\) pair.
Due to degeneration in the absence of external fields, the coupling can
be propagated very long to lower \(l\) states, which explosed the basis.
This was verified by enlarging the cut off of \(l\) and each time the
value changed dramatically. So let's keep a constant magnetic field of
\(\sim 10\) G.

    \begin{Verbatim}[commandchars=\\\{\}]
{\color{incolor}In [{\color{incolor}41}]:} \PY{c}{\PYZsh{} Define levels builtins to globalize the parameters}
         \PY{k+kn}{import} \PY{n+nn}{builtins}
         \PY{n}{builtins}\PY{o}{.}\PY{n}{n1} \PY{o}{=} \PY{l+m+mi}{60}
         \PY{n}{builtins}\PY{o}{.}\PY{n}{l1} \PY{o}{=} \PY{n}{n1}\PY{o}{\PYZhy{}}\PY{l+m+mi}{1}
         \PY{n}{builtins}\PY{o}{.}\PY{n}{m1} \PY{o}{=} \PY{n}{l1}
         
         \PY{n}{builtins}\PY{o}{.}\PY{n}{n2} \PY{o}{=} \PY{l+m+mi}{60}
         \PY{n}{builtins}\PY{o}{.}\PY{n}{l2} \PY{o}{=} \PY{n}{n2}\PY{o}{\PYZhy{}}\PY{l+m+mi}{1}     
         \PY{n}{builtins}\PY{o}{.}\PY{n}{m2} \PY{o}{=} \PY{n}{l2}
         
         \PY{n}{builtins}\PY{o}{.}\PY{n}{Bfield} \PY{o}{=} \PY{l+m+mf}{10.e\PYZhy{}4} \PY{c}{\PYZsh{} Magnetic field from experiment 1 Teslta = 10*4 Gauss}
         \PY{n}{builtins}\PY{o}{.}\PY{n}{theta} \PY{o}{=} \PY{l+m+mi}{0}\PY{o}{*}\PY{n}{pi}\PY{o}{/}\PY{l+m+mi}{2} \PY{c}{\PYZsh{} angle between magnetic field (quantization axis) and atom pair}
\end{Verbatim}

    Set some criterion for the program but mainly not in use now

    \begin{Verbatim}[commandchars=\\\{\}]
{\color{incolor}In [{\color{incolor}42}]:} \PY{c}{\PYZsh{} Setup criterion}
         \PY{n}{delta\PYZus{}n\PYZus{}max} \PY{o}{=} \PY{l+m+mi}{6}
         \PY{n}{l\PYZus{}max} \PY{o}{=} \PY{l+m+mi}{2}
         \PY{n}{Choice} \PY{o}{=} \PY{l+m+mf}{1e7}\PY{c}{\PYZsh{} cut off energy for 1st order term, in Hz}
         \PY{n}{builtins}\PY{o}{.}\PY{n}{R\PYZus{}test} \PY{o}{=} \PY{l+m+mf}{1e\PYZhy{}6}
         \PY{n}{Choice2} \PY{o}{=} \PY{l+m+mi}{10}\PY{o}{*} \PY{n}{Choice}
         
         \PY{n}{builtins}\PY{o}{.}\PY{n}{Choice\PYZus{}F} \PY{o}{=} \PY{l+m+mf}{1e\PYZhy{}1} \PY{c}{\PYZsh{} cut off for Stark shift}
\end{Verbatim}

    \subsection{B field aligned with pair of
atoms}\label{b-field-aligned-with-pair-of-atoms}

\subsubsection{No electric field}\label{no-electric-field}

    Define electric field and set up configuration and launch the
calculation from VdW file

    \begin{Verbatim}[commandchars=\\\{\}]
{\color{incolor}In [{\color{incolor}18}]:} \PY{n}{builtins}\PY{o}{.}\PY{n}{Ffield} \PY{o}{=} \PY{l+m+mi}{0}\PY{o}{*}\PY{l+m+mf}{1e\PYZhy{}1} \PY{c}{\PYZsh{} V/cm}
         \PY{n}{builtins}\PY{o}{.}\PY{n}{theta\PYZus{}F} \PY{o}{=} \PY{l+m+mf}{0.001}\PY{o}{*}\PY{n}{pi}\PY{o}{/}\PY{l+m+mi}{2} \PY{c}{\PYZsh{} angle between F field and magnetic field}
         \PY{n}{builtins}\PY{o}{.}\PY{n}{phi\PYZus{}F} \PY{o}{=} \PY{l+m+mf}{0.001}\PY{o}{*}\PY{n}{pi}\PY{o}{/}\PY{l+m+mi}{2} \PY{c}{\PYZsh{} angle between F\PYZus{}field and magnetic field , atom pair plan}
         \PY{o}{\PYZpc{}}\PY{k}{run} \PY{o}{\PYZhy{}}\PY{n}{n} \PY{n}{cal\PYZus{}VdW}\PY{o}{.}\PY{n}{py}
\end{Verbatim}

    \begin{Verbatim}[commandchars=\\\{\}]
atom 60C and atom 60C
theta = 0.0 deg
B\_field = 10.0 G
F\_field = 0.0 V/cm
theta\_F = 0.09 deg
phi\_F = 0.09 deg
Matrix size: 482
    \end{Verbatim}

    \begin{Verbatim}[commandchars=\\\{\}]
C:\textbackslash{}Users\textbackslash{}r14\textbackslash{}Documents\textbackslash{}GitHub\textbackslash{}test\textbackslash{}python\textbackslash{}cal\_VdW.py:271: ComplexWarning: Casting complex values to real discards the imaginary part
  out\_egr[i] , out\_vector[i] = np.linalg.eigh(EI + 1e18*V\_VdW* coef/(elm**3) + coef\_F*(V\_Stark1 + V\_Stark2))
    \end{Verbatim}

    \begin{center}
    \adjustimage{max size={0.9\linewidth}{0.9\paperheight}}{VdW for circular Rydberg atoms_files/VdW for circular Rydberg atoms_7_2.png}
    \end{center}
    { \hspace*{\fill} \\}
    
    \begin{center}
    \adjustimage{max size={0.9\linewidth}{0.9\paperheight}}{VdW for circular Rydberg atoms_files/VdW for circular Rydberg atoms_7_3.png}
    \end{center}
    { \hspace*{\fill} \\}
    
    \begin{Verbatim}[commandchars=\\\{\}]
C6 = [ 49.5525243] GHz.um\^{}6
At R = 2.4740253954047375 um, prop > 5\% are:
pair of atom 59C and atom 61C, at index 144
pair of atom 61C and atom 59C, at index 145
pair of atom 60C and atom 60C, at index 181
    \end{Verbatim}

    A Van der Waals potential fits well with the result up to 3 \(\mu\)m
where the levels mixed (dipole interactions comparable to Zeeman shifts)
in which the \(|60C60C\rangle\) couple quasi resonance with
\(|60E60E\rangle\) which leads to \(1/R^3\) interaction. Last figure
show the contribution from \(|60C60C\rangle\) where from \(\sim 5\mu\)m,
it is nolonger pure

\(C_6 = 49.55\) GHz\(.\mu\)m\(^6\)

    \subsubsection{Electric field, aligned with magnetic field, atom pair
aligned to magnetic
field}\label{electric-field-aligned-with-magnetic-field-atom-pair-aligned-to-magnetic-field}

    \begin{Verbatim}[commandchars=\\\{\}]
{\color{incolor}In [{\color{incolor}19}]:} \PY{n}{builtins}\PY{o}{.}\PY{n}{Ffield} \PY{o}{=} \PY{l+m+mf}{1.e\PYZhy{}1} \PY{c}{\PYZsh{} V/cm}
         \PY{n}{builtins}\PY{o}{.}\PY{n}{theta\PYZus{}F} \PY{o}{=} \PY{l+m+mf}{0.}\PY{o}{*}\PY{n}{pi}\PY{o}{/}\PY{l+m+mi}{2} \PY{c}{\PYZsh{} angle between F field and magnetic field}
         \PY{n}{builtins}\PY{o}{.}\PY{n}{phi\PYZus{}F} \PY{o}{=} \PY{l+m+mf}{0.}\PY{o}{*}\PY{n}{pi}\PY{o}{/}\PY{l+m+mi}{2} \PY{c}{\PYZsh{} angle between F\PYZus{}field and magnetic field , atom pair plan}
         \PY{o}{\PYZpc{}}\PY{k}{run} \PY{o}{\PYZhy{}}\PY{n}{n} \PY{n}{cal\PYZus{}VdW}\PY{o}{.}\PY{n}{py}
\end{Verbatim}

    \begin{Verbatim}[commandchars=\\\{\}]
atom 60C and atom 60C
theta = 0.0 deg
B\_field = 10.0 G
F\_field = 0.1 V/cm
theta\_F = 0.0 deg
phi\_F = 0.0 deg
Matrix size: 482
    \end{Verbatim}

    \begin{Verbatim}[commandchars=\\\{\}]
C:\textbackslash{}Users\textbackslash{}r14\textbackslash{}Documents\textbackslash{}GitHub\textbackslash{}test\textbackslash{}python\textbackslash{}cal\_VdW.py:271: ComplexWarning: Casting complex values to real discards the imaginary part
  out\_egr[i] , out\_vector[i] = np.linalg.eigh(EI + 1e18*V\_VdW* coef/(elm**3) + coef\_F*(V\_Stark1 + V\_Stark2))
    \end{Verbatim}

    \begin{center}
    \adjustimage{max size={0.9\linewidth}{0.9\paperheight}}{VdW for circular Rydberg atoms_files/VdW for circular Rydberg atoms_10_2.png}
    \end{center}
    { \hspace*{\fill} \\}
    
    \begin{center}
    \adjustimage{max size={0.9\linewidth}{0.9\paperheight}}{VdW for circular Rydberg atoms_files/VdW for circular Rydberg atoms_10_3.png}
    \end{center}
    { \hspace*{\fill} \\}
    
    \begin{Verbatim}[commandchars=\\\{\}]
C6 = [ 49.5523487] GHz.um\^{}6
At R = 2.4740253954047375 um, prop > 5\% are:
pair of atom 59C and atom 61C, at index 144
pair of atom 61C and atom 59C, at index 145
pair of atom 60C and atom 60C, at index 181
    \end{Verbatim}

    \(C_6 = 49.55\) GHz\(\mu\)m\(^6\)

More less the same with the above result, which is easily understood
from the configuration. More than that the propagation of coupling has
very small effect as the Hamiltonian only couple pair states which
conserve \(M = m1 + m2\). Now let's tilt the atomic pair to
perpendicular to the field.

\subsubsection{F field perpendicular to B field which is aligned to pair
of
atoms}\label{f-field-perpendicular-to-b-field-which-is-aligned-to-pair-of-atoms}

    \begin{Verbatim}[commandchars=\\\{\}]
{\color{incolor}In [{\color{incolor}20}]:} \PY{n}{builtins}\PY{o}{.}\PY{n}{theta} \PY{o}{=} \PY{l+m+mi}{0}\PY{o}{*}\PY{n}{pi}\PY{o}{/}\PY{l+m+mi}{2} \PY{c}{\PYZsh{} angle between magnetic field (quantization axis) and atom pair}
         \PY{n}{builtins}\PY{o}{.}\PY{n}{Ffield} \PY{o}{=} \PY{l+m+mf}{1.e\PYZhy{}2} \PY{c}{\PYZsh{} V/cm}
         \PY{n}{builtins}\PY{o}{.}\PY{n}{theta\PYZus{}F} \PY{o}{=} \PY{l+m+mi}{1}\PY{o}{*}\PY{n}{pi}\PY{o}{/}\PY{l+m+mi}{2} \PY{c}{\PYZsh{} angle between F field and magnetic field}
         \PY{n}{builtins}\PY{o}{.}\PY{n}{phi\PYZus{}F} \PY{o}{=} \PY{l+m+mf}{0.}\PY{o}{*}\PY{n}{pi}\PY{o}{/}\PY{l+m+mi}{2} \PY{c}{\PYZsh{} angle between F\PYZus{}field and magnetic field , atom pair plan}
         \PY{o}{\PYZpc{}}\PY{k}{run} \PY{o}{\PYZhy{}}\PY{n}{n} \PY{n}{cal\PYZus{}VdW}\PY{o}{.}\PY{n}{py}
\end{Verbatim}

    \begin{Verbatim}[commandchars=\\\{\}]
atom 60C and atom 60C
theta = 0.0 deg
B\_field = 10.0 G
F\_field = 0.01 V/cm
theta\_F = 90.0 deg
phi\_F = 0.0 deg
Matrix size: 482
    \end{Verbatim}

    \begin{Verbatim}[commandchars=\\\{\}]
C:\textbackslash{}Users\textbackslash{}r14\textbackslash{}Documents\textbackslash{}GitHub\textbackslash{}test\textbackslash{}python\textbackslash{}cal\_VdW.py:271: ComplexWarning: Casting complex values to real discards the imaginary part
  out\_egr[i] , out\_vector[i] = np.linalg.eigh(EI + 1e18*V\_VdW* coef/(elm**3) + coef\_F*(V\_Stark1 + V\_Stark2))
    \end{Verbatim}

    \begin{center}
    \adjustimage{max size={0.9\linewidth}{0.9\paperheight}}{VdW for circular Rydberg atoms_files/VdW for circular Rydberg atoms_12_2.png}
    \end{center}
    { \hspace*{\fill} \\}
    
    \begin{center}
    \adjustimage{max size={0.9\linewidth}{0.9\paperheight}}{VdW for circular Rydberg atoms_files/VdW for circular Rydberg atoms_12_3.png}
    \end{center}
    { \hspace*{\fill} \\}
    
    \begin{Verbatim}[commandchars=\\\{\}]
C4 = [ 2.49046573] GHz.um\^{}4
At R = 29.999999999999996 um, prop > 5\% are:
pair of atom 60.0, 58.0, 58.0 and atom 60C, at index 177
pair of atom 60C and atom 60.0, 58.0, 58.0, at index 179
pair of atom 60C and atom 60C, at index 181
    \end{Verbatim}

    It is complicated due to Stark effect. Let's try with a smaller F field

    \begin{Verbatim}[commandchars=\\\{\}]
{\color{incolor}In [{\color{incolor}21}]:} \PY{n}{builtins}\PY{o}{.}\PY{n}{theta} \PY{o}{=} \PY{l+m+mi}{0}\PY{o}{*}\PY{n}{pi}\PY{o}{/}\PY{l+m+mi}{2} \PY{c}{\PYZsh{} angle between magnetic field (quantization axis) and atom pair}
         \PY{n}{builtins}\PY{o}{.}\PY{n}{Ffield} \PY{o}{=} \PY{l+m+mf}{5.e\PYZhy{}3} \PY{c}{\PYZsh{} V/cm}
         \PY{n}{builtins}\PY{o}{.}\PY{n}{theta\PYZus{}F} \PY{o}{=} \PY{l+m+mi}{1}\PY{o}{*}\PY{n}{pi}\PY{o}{/}\PY{l+m+mi}{2} \PY{c}{\PYZsh{} angle between F field and magnetic field}
         \PY{n}{builtins}\PY{o}{.}\PY{n}{phi\PYZus{}F} \PY{o}{=} \PY{l+m+mf}{0.}\PY{o}{*}\PY{n}{pi}\PY{o}{/}\PY{l+m+mi}{2} \PY{c}{\PYZsh{} angle between F\PYZus{}field and magnetic field , atom pair plan}
         \PY{o}{\PYZpc{}}\PY{k}{run} \PY{o}{\PYZhy{}}\PY{n}{n} \PY{n}{cal\PYZus{}VdW}\PY{o}{.}\PY{n}{py}
\end{Verbatim}

    \begin{Verbatim}[commandchars=\\\{\}]
atom 60C and atom 60C
theta = 0.0 deg
B\_field = 10.0 G
F\_field = 0.005 V/cm
theta\_F = 90.0 deg
phi\_F = 0.0 deg
Matrix size: 482
    \end{Verbatim}

    \begin{Verbatim}[commandchars=\\\{\}]
C:\textbackslash{}Users\textbackslash{}r14\textbackslash{}Documents\textbackslash{}GitHub\textbackslash{}test\textbackslash{}python\textbackslash{}cal\_VdW.py:271: ComplexWarning: Casting complex values to real discards the imaginary part
  out\_egr[i] , out\_vector[i] = np.linalg.eigh(EI + 1e18*V\_VdW* coef/(elm**3) + coef\_F*(V\_Stark1 + V\_Stark2))
    \end{Verbatim}

    \begin{center}
    \adjustimage{max size={0.9\linewidth}{0.9\paperheight}}{VdW for circular Rydberg atoms_files/VdW for circular Rydberg atoms_14_2.png}
    \end{center}
    { \hspace*{\fill} \\}
    
    \begin{center}
    \adjustimage{max size={0.9\linewidth}{0.9\paperheight}}{VdW for circular Rydberg atoms_files/VdW for circular Rydberg atoms_14_3.png}
    \end{center}
    { \hspace*{\fill} \\}
    
    \begin{Verbatim}[commandchars=\\\{\}]
C5 = [ 10.20892795] GHz.um\^{}5
At R = 8.325633024798213 um, prop > 5\% are:
pair of atom 60.0, 58.0, 58.0 and atom 60C, at index 177
pair of atom 60C and atom 60.0, 58.0, 58.0, at index 179
pair of atom 60C and atom 60C, at index 181
    \end{Verbatim}

    There is a change from \(1/R^3\) to \(1/R^6\) which is not yet
understood \textgreater{}``\textless{}

Let's consider the case B // pair. F makes small angle. \#\#\# F
\textasciitilde{}// B // atom pair\#\#\#

    \begin{Verbatim}[commandchars=\\\{\}]
{\color{incolor}In [{\color{incolor}43}]:} \PY{n}{builtins}\PY{o}{.}\PY{n}{theta} \PY{o}{=} \PY{l+m+mi}{0}\PY{o}{*}\PY{n}{pi}\PY{o}{/}\PY{l+m+mi}{2} \PY{c}{\PYZsh{} angle between magnetic field (quantization axis) and atom pair}
         \PY{n}{builtins}\PY{o}{.}\PY{n}{Ffield} \PY{o}{=} \PY{l+m+mf}{10.e\PYZhy{}3} \PY{c}{\PYZsh{} V/cm}
         \PY{n}{builtins}\PY{o}{.}\PY{n}{theta\PYZus{}F} \PY{o}{=} \PY{l+m+mf}{0.1}\PY{o}{*}\PY{n}{pi}\PY{o}{/}\PY{l+m+mi}{2} \PY{c}{\PYZsh{} angle between F field and magnetic field}
         \PY{n}{builtins}\PY{o}{.}\PY{n}{phi\PYZus{}F} \PY{o}{=} \PY{l+m+mf}{0.}\PY{o}{*}\PY{n}{pi}\PY{o}{/}\PY{l+m+mi}{2} \PY{c}{\PYZsh{} angle between F\PYZus{}field and magnetic field , atom pair plan}
         \PY{o}{\PYZpc{}}\PY{k}{run} \PY{o}{\PYZhy{}}\PY{n}{n} \PY{n}{cal\PYZus{}VdW}\PY{o}{.}\PY{n}{py}
\end{Verbatim}

    \begin{Verbatim}[commandchars=\\\{\}]
atom 60C and atom 60C
theta = 0.0 deg
B\_field = 10.0 G
F\_field = 0.01 V/cm
theta\_F = 9.0 deg
phi\_F = 0.0 deg
Matrix size: 482
    \end{Verbatim}

    \begin{Verbatim}[commandchars=\\\{\}]
C:\textbackslash{}Users\textbackslash{}r14\textbackslash{}Documents\textbackslash{}GitHub\textbackslash{}test\textbackslash{}python\textbackslash{}cal\_VdW.py:271: ComplexWarning: Casting complex values to real discards the imaginary part
  out\_egr[i] , out\_vector[i] = np.linalg.eigh(EI + 1e18*V\_VdW* coef/(elm**3) + coef\_F*(V\_Stark1 + V\_Stark2))
    \end{Verbatim}

    \begin{center}
    \adjustimage{max size={0.9\linewidth}{0.9\paperheight}}{VdW for circular Rydberg atoms_files/VdW for circular Rydberg atoms_16_2.png}
    \end{center}
    { \hspace*{\fill} \\}
    
    \begin{center}
    \adjustimage{max size={0.9\linewidth}{0.9\paperheight}}{VdW for circular Rydberg atoms_files/VdW for circular Rydberg atoms_16_3.png}
    \end{center}
    { \hspace*{\fill} \\}
    
    \begin{Verbatim}[commandchars=\\\{\}]
C6 = [ 50.73317894] GHz.um\^{}6
At R = 3.482238803984437 um, prop > 5\% are:
pair of atom 60.0, 58.0, 58.0 and atom 60C, at index 177
pair of atom 60C and atom 60.0, 58.0, 58.0, at index 179
pair of atom 60C and atom 60C, at index 181
    \end{Verbatim}

    \begin{Verbatim}[commandchars=\\\{\}]
{\color{incolor}In [{\color{incolor}60}]:} \PY{n}{subplot}\PY{p}{(}\PY{l+m+mi}{2}\PY{p}{,}\PY{l+m+mi}{2}\PY{p}{,}\PY{l+m+mi}{1}\PY{p}{)}\PY{p}{;}\PY{n}{semilogx}\PY{p}{(}\PY{n}{R}\PY{p}{,} \PY{n}{out\PYZus{}egr}\PY{p}{[}\PY{p}{:}\PY{p}{,}\PY{l+m+mi}{177}\PY{p}{:}\PY{l+m+mi}{182}\PY{p}{]}\PY{p}{)}\PY{p}{;}\PY{n}{ylim}\PY{p}{(}\PY{n}{out\PYZus{}egr1}\PY{p}{[}\PY{o}{\PYZhy{}}\PY{l+m+mi}{1}\PY{p}{]}\PY{o}{\PYZhy{}}\PY{l+m+mf}{0.05}\PY{p}{,} \PY{n}{out\PYZus{}egr1}\PY{p}{[}\PY{o}{\PYZhy{}}\PY{l+m+mi}{1}\PY{p}{]}\PY{o}{+}\PY{l+m+mf}{0.05}\PY{p}{)}\PY{p}{;}
         \PY{n}{subplot}\PY{p}{(}\PY{l+m+mi}{2}\PY{p}{,}\PY{l+m+mi}{2}\PY{p}{,}\PY{l+m+mi}{2}\PY{p}{)}\PY{p}{;}\PY{n}{imshow}\PY{p}{(}\PY{n}{out\PYZus{}vector}\PY{p}{[}\PY{o}{\PYZhy{}}\PY{l+m+mi}{1}\PY{p}{,}\PY{l+m+mi}{177}\PY{p}{:}\PY{l+m+mi}{182}\PY{p}{,}\PY{l+m+mi}{177}\PY{p}{:}\PY{l+m+mi}{182}\PY{p}{]}\PY{o}{*}\PY{o}{*}\PY{l+m+mi}{2}\PY{p}{)}\PY{p}{;}\PY{n}{colorbar}\PY{p}{(}\PY{p}{)}
         \PY{n}{subplot}\PY{p}{(}\PY{l+m+mi}{2}\PY{p}{,}\PY{l+m+mi}{2}\PY{p}{,}\PY{l+m+mi}{3}\PY{p}{)}\PY{p}{;}\PY{n}{plot}\PY{p}{(}\PY{n}{out\PYZus{}vector}\PY{p}{[}\PY{o}{\PYZhy{}}\PY{l+m+mi}{1}\PY{p}{,} \PY{l+m+mi}{177}\PY{p}{,}\PY{l+m+mi}{177}\PY{p}{:}\PY{l+m+mi}{182}\PY{p}{]}\PY{o}{*}\PY{o}{*}\PY{l+m+mi}{2}\PY{p}{)}
         \PY{n}{subplot}\PY{p}{(}\PY{l+m+mi}{2}\PY{p}{,}\PY{l+m+mi}{2}\PY{p}{,}\PY{l+m+mi}{4}\PY{p}{)}\PY{p}{;}\PY{n}{plot}\PY{p}{(}\PY{n}{out\PYZus{}vector}\PY{p}{[}\PY{o}{\PYZhy{}}\PY{l+m+mi}{1}\PY{p}{,} \PY{l+m+mi}{179}\PY{p}{,}\PY{l+m+mi}{177}\PY{p}{:}\PY{l+m+mi}{182}\PY{p}{]}\PY{o}{*}\PY{o}{*}\PY{l+m+mi}{2}\PY{p}{)}
         \PY{n}{show}\PY{p}{(}\PY{p}{)}
         \PY{k}{print}\PY{p}{(}\PY{n}{Union\PYZus{}list}\PY{p}{[}\PY{l+m+mi}{177}\PY{p}{:}\PY{l+m+mi}{181}\PY{p}{]}\PY{p}{)}
\end{Verbatim}

    \begin{center}
    \adjustimage{max size={0.9\linewidth}{0.9\paperheight}}{VdW for circular Rydberg atoms_files/VdW for circular Rydberg atoms_17_0.png}
    \end{center}
    { \hspace*{\fill} \\}
    
    \begin{Verbatim}[commandchars=\\\{\}]
[atom 60.0, 58.0, 58.0 and atom 60C, atom 60.0, 59.0, 58.0 and atom 60C, atom 60C and atom 60.0, 58.0, 58.0, atom 60C and atom 60.0, 59.0, 58.0]
    \end{Verbatim}

    \subsection{B field perpendicular to pair of
atoms}\label{b-field-perpendicular-to-pair-of-atoms}

\subsubsection{No electric field}\label{no-electric-field}

    \begin{Verbatim}[commandchars=\\\{\}]
{\color{incolor}In [{\color{incolor}22}]:} \PY{n}{builtins}\PY{o}{.}\PY{n}{theta} \PY{o}{=} \PY{n}{pi}\PY{o}{/}\PY{l+m+mi}{2} \PY{c}{\PYZsh{} angle between magnetic field (quantization axis) and atom pair}
         \PY{n}{builtins}\PY{o}{.}\PY{n}{Ffield} \PY{o}{=} \PY{l+m+mi}{0}\PY{o}{*}\PY{l+m+mf}{1.e\PYZhy{}1} \PY{c}{\PYZsh{} V/cm}
         \PY{n}{builtins}\PY{o}{.}\PY{n}{theta\PYZus{}F} \PY{o}{=} \PY{l+m+mf}{0.}\PY{o}{*}\PY{n}{pi}\PY{o}{/}\PY{l+m+mi}{2} \PY{c}{\PYZsh{} angle between F field and magnetic field}
         \PY{n}{builtins}\PY{o}{.}\PY{n}{phi\PYZus{}F} \PY{o}{=} \PY{l+m+mf}{0.}\PY{o}{*}\PY{n}{pi}\PY{o}{/}\PY{l+m+mi}{2} \PY{c}{\PYZsh{} angle between F\PYZus{}field and magnetic field , atom pair plan}
         \PY{o}{\PYZpc{}}\PY{k}{run} \PY{o}{\PYZhy{}}\PY{n}{n} \PY{n}{cal\PYZus{}VdW}\PY{o}{.}\PY{n}{py}
\end{Verbatim}

    \begin{Verbatim}[commandchars=\\\{\}]
atom 60C and atom 60C
theta = 90.0 deg
B\_field = 10.0 G
F\_field = 0.0 V/cm
theta\_F = 0.0 deg
phi\_F = 0.0 deg
Matrix size: 482
    \end{Verbatim}

    \begin{Verbatim}[commandchars=\\\{\}]
C:\textbackslash{}Users\textbackslash{}r14\textbackslash{}Documents\textbackslash{}GitHub\textbackslash{}test\textbackslash{}python\textbackslash{}cal\_VdW.py:271: ComplexWarning: Casting complex values to real discards the imaginary part
  out\_egr[i] , out\_vector[i] = np.linalg.eigh(EI + 1e18*V\_VdW* coef/(elm**3) + coef\_F*(V\_Stark1 + V\_Stark2))
    \end{Verbatim}

    \begin{center}
    \adjustimage{max size={0.9\linewidth}{0.9\paperheight}}{VdW for circular Rydberg atoms_files/VdW for circular Rydberg atoms_19_2.png}
    \end{center}
    { \hspace*{\fill} \\}
    
    \begin{center}
    \adjustimage{max size={0.9\linewidth}{0.9\paperheight}}{VdW for circular Rydberg atoms_files/VdW for circular Rydberg atoms_19_3.png}
    \end{center}
    { \hspace*{\fill} \\}
    
    \begin{Verbatim}[commandchars=\\\{\}]
C6 = [ 30.76966515] GHz.um\^{}6
At R = 4.8182599654372344 um, prop > 5\% are:
pair of atom 60.0, 58.0, 58.0 and atom 60.0, 58.0, 58.0, at index 173
pair of atom 60C and atom 60C, at index 181
    \end{Verbatim}

    The coupling is more complicated as much more levels contribute. The
effect of degeneration is much more crutial. Van der Waals still fits as
magnetic field lift up the degeneracies.

\(C_6 = 30.77\) GHz.\(\mu m^6\). The anti crossing is at about
\(3 \mu\)m.

\subsubsection{Weak electric field}\label{weak-electric-field}

    \begin{Verbatim}[commandchars=\\\{\}]
{\color{incolor}In [{\color{incolor}23}]:} \PY{n}{builtins}\PY{o}{.}\PY{n}{theta} \PY{o}{=} \PY{n}{pi}\PY{o}{/}\PY{l+m+mi}{2} \PY{c}{\PYZsh{} angle between magnetic field (quantization axis) and atom pair}
         \PY{n}{builtins}\PY{o}{.}\PY{n}{Ffield} \PY{o}{=} \PY{l+m+mf}{1.e\PYZhy{}1} \PY{c}{\PYZsh{} V/cm}
         \PY{n}{builtins}\PY{o}{.}\PY{n}{theta\PYZus{}F} \PY{o}{=} \PY{l+m+mf}{0.}\PY{o}{*}\PY{n}{pi}\PY{o}{/}\PY{l+m+mi}{2} \PY{c}{\PYZsh{} angle between F field and magnetic field}
         \PY{n}{builtins}\PY{o}{.}\PY{n}{phi\PYZus{}F} \PY{o}{=} \PY{l+m+mf}{0.}\PY{o}{*}\PY{n}{pi}\PY{o}{/}\PY{l+m+mi}{2} \PY{c}{\PYZsh{} angle between F\PYZus{}field and magnetic field , atom pair plan}
         \PY{o}{\PYZpc{}}\PY{k}{run} \PY{o}{\PYZhy{}}\PY{n}{n} \PY{n}{cal\PYZus{}VdW}\PY{o}{.}\PY{n}{py}
\end{Verbatim}

    \begin{Verbatim}[commandchars=\\\{\}]
atom 60C and atom 60C
theta = 90.0 deg
B\_field = 10.0 G
F\_field = 0.1 V/cm
theta\_F = 0.0 deg
phi\_F = 0.0 deg
Matrix size: 482
    \end{Verbatim}

    \begin{Verbatim}[commandchars=\\\{\}]
C:\textbackslash{}Users\textbackslash{}r14\textbackslash{}Documents\textbackslash{}GitHub\textbackslash{}test\textbackslash{}python\textbackslash{}cal\_VdW.py:271: ComplexWarning: Casting complex values to real discards the imaginary part
  out\_egr[i] , out\_vector[i] = np.linalg.eigh(EI + 1e18*V\_VdW* coef/(elm**3) + coef\_F*(V\_Stark1 + V\_Stark2))
    \end{Verbatim}

    \begin{center}
    \adjustimage{max size={0.9\linewidth}{0.9\paperheight}}{VdW for circular Rydberg atoms_files/VdW for circular Rydberg atoms_21_2.png}
    \end{center}
    { \hspace*{\fill} \\}
    
    \begin{center}
    \adjustimage{max size={0.9\linewidth}{0.9\paperheight}}{VdW for circular Rydberg atoms_files/VdW for circular Rydberg atoms_21_3.png}
    \end{center}
    { \hspace*{\fill} \\}
    
    \begin{Verbatim}[commandchars=\\\{\}]
C6 = [ 54.42595878] GHz.um\^{}6
At R = 6.226314108246262 um, prop > 5\% are:
pair of atom 60.0, 58.0, 58.0 and atom 60.0, 58.0, 58.0, at index 173
pair of atom 60C and atom 60C, at index 181
    \end{Verbatim}

    The crossing point is moved to larger distances as the Stark effect
starts to take over the Zeeman splitting.

\(C_6 = 54.4\) GHz.\(\mu m^6\) up to \(\sim 5 \mu\)m. To \(\sim 10\mu\)m
the pair is still more less pure circular. Increase \(l\) terms yields
\(C_6 = 53.25\) GHz.\(\mu m^6\). If we put in a little bit stronger
field, levels are mixed up and the mess happens (verified for 200 mV/cm)

\subsubsection{Strong electric field}\label{strong-electric-field}

    \begin{Verbatim}[commandchars=\\\{\}]
{\color{incolor}In [{\color{incolor}24}]:} \PY{n}{builtins}\PY{o}{.}\PY{n}{theta} \PY{o}{=} \PY{n}{pi}\PY{o}{/}\PY{l+m+mi}{2} \PY{c}{\PYZsh{} angle between magnetic field (quantization axis) and atom pair}
         \PY{n}{builtins}\PY{o}{.}\PY{n}{Ffield} \PY{o}{=} \PY{l+m+mf}{1.e\PYZhy{}0} \PY{c}{\PYZsh{} V/cm}
         \PY{n}{builtins}\PY{o}{.}\PY{n}{theta\PYZus{}F} \PY{o}{=} \PY{l+m+mf}{0.}\PY{o}{*}\PY{n}{pi}\PY{o}{/}\PY{l+m+mi}{2} \PY{c}{\PYZsh{} angle between F field and magnetic field}
         \PY{n}{builtins}\PY{o}{.}\PY{n}{phi\PYZus{}F} \PY{o}{=} \PY{l+m+mf}{0.}\PY{o}{*}\PY{n}{pi}\PY{o}{/}\PY{l+m+mi}{2} \PY{c}{\PYZsh{} angle between F\PYZus{}field and magnetic field , atom pair plan}
         \PY{o}{\PYZpc{}}\PY{k}{run} \PY{o}{\PYZhy{}}\PY{n}{n} \PY{n}{cal\PYZus{}VdW}\PY{o}{.}\PY{n}{py}
\end{Verbatim}

    \begin{Verbatim}[commandchars=\\\{\}]
atom 60C and atom 60C
theta = 90.0 deg
B\_field = 10.0 G
F\_field = 1.0 V/cm
theta\_F = 0.0 deg
phi\_F = 0.0 deg
Matrix size: 482
    \end{Verbatim}

    \begin{Verbatim}[commandchars=\\\{\}]
C:\textbackslash{}Users\textbackslash{}r14\textbackslash{}Documents\textbackslash{}GitHub\textbackslash{}test\textbackslash{}python\textbackslash{}cal\_VdW.py:271: ComplexWarning: Casting complex values to real discards the imaginary part
  out\_egr[i] , out\_vector[i] = np.linalg.eigh(EI + 1e18*V\_VdW* coef/(elm**3) + coef\_F*(V\_Stark1 + V\_Stark2))
    \end{Verbatim}

    \begin{center}
    \adjustimage{max size={0.9\linewidth}{0.9\paperheight}}{VdW for circular Rydberg atoms_files/VdW for circular Rydberg atoms_23_2.png}
    \end{center}
    { \hspace*{\fill} \\}
    
    \begin{center}
    \adjustimage{max size={0.9\linewidth}{0.9\paperheight}}{VdW for circular Rydberg atoms_files/VdW for circular Rydberg atoms_23_3.png}
    \end{center}
    { \hspace*{\fill} \\}
    
    \begin{Verbatim}[commandchars=\\\{\}]
C6 = [ 21.50986621] GHz.um\^{}6
At R = 3.7929066150308723 um, prop > 5\% are:
pair of atom 60.0, 59.0, 58.0 and atom 60.0, 59.0, 58.0, at index 176
pair of atom 60C and atom 60C, at index 181
    \end{Verbatim}

    In fact the levels are mixed up in a complicated way. The
\(|60C60C\rangle\) is no longer on the top, isolated from the other
eliptical states. We have to introduce more terms into our basis. The
result here is \textbf{not trustable}.

\subsubsection{Effect of electric field
alignment}\label{effect-of-electric-field-alignment}

\paragraph{In plan of B field and atomic
pair}\label{in-plan-of-b-field-and-atomic-pair}

    \begin{Verbatim}[commandchars=\\\{\}]
{\color{incolor}In [{\color{incolor}25}]:} \PY{n}{builtins}\PY{o}{.}\PY{n}{theta} \PY{o}{=} \PY{n}{pi}\PY{o}{/}\PY{l+m+mi}{2} \PY{c}{\PYZsh{} angle between magnetic field (quantization axis) and atom pair}
         \PY{n}{builtins}\PY{o}{.}\PY{n}{Ffield} \PY{o}{=} \PY{l+m+mf}{1.e\PYZhy{}2} \PY{c}{\PYZsh{} V/cm}
         \PY{n}{builtins}\PY{o}{.}\PY{n}{theta\PYZus{}F} \PY{o}{=} \PY{l+m+mf}{0.1}\PY{o}{*}\PY{n}{pi}\PY{o}{/}\PY{l+m+mi}{2} \PY{c}{\PYZsh{} angle between F field and magnetic field}
         \PY{n}{builtins}\PY{o}{.}\PY{n}{phi\PYZus{}F} \PY{o}{=} \PY{l+m+mf}{0.}\PY{o}{*}\PY{n}{pi}\PY{o}{/}\PY{l+m+mi}{2} \PY{c}{\PYZsh{} angle between F\PYZus{}field and magnetic field , atom pair plan}
         \PY{o}{\PYZpc{}}\PY{k}{run} \PY{o}{\PYZhy{}}\PY{n}{n} \PY{n}{cal\PYZus{}VdW}\PY{o}{.}\PY{n}{py}
\end{Verbatim}

    \begin{Verbatim}[commandchars=\\\{\}]
atom 60C and atom 60C
theta = 90.0 deg
B\_field = 10.0 G
F\_field = 0.01 V/cm
theta\_F = 9.0 deg
phi\_F = 0.0 deg
Matrix size: 482
    \end{Verbatim}

    \begin{Verbatim}[commandchars=\\\{\}]
C:\textbackslash{}Users\textbackslash{}r14\textbackslash{}Documents\textbackslash{}GitHub\textbackslash{}test\textbackslash{}python\textbackslash{}cal\_VdW.py:271: ComplexWarning: Casting complex values to real discards the imaginary part
  out\_egr[i] , out\_vector[i] = np.linalg.eigh(EI + 1e18*V\_VdW* coef/(elm**3) + coef\_F*(V\_Stark1 + V\_Stark2))
    \end{Verbatim}

    \begin{center}
    \adjustimage{max size={0.9\linewidth}{0.9\paperheight}}{VdW for circular Rydberg atoms_files/VdW for circular Rydberg atoms_25_2.png}
    \end{center}
    { \hspace*{\fill} \\}
    
    \begin{center}
    \adjustimage{max size={0.9\linewidth}{0.9\paperheight}}{VdW for circular Rydberg atoms_files/VdW for circular Rydberg atoms_25_3.png}
    \end{center}
    { \hspace*{\fill} \\}
    
    \begin{Verbatim}[commandchars=\\\{\}]
C6 = [ 29.24132198] GHz.um\^{}6
At R = 4.8182599654372344 um, prop > 5\% are:
pair of atom 60.0, 58.0, 58.0 and atom 60.0, 58.0, 58.0, at index 173
pair of atom 60C and atom 60C, at index 181
    \end{Verbatim}

    \(C_6 \sim 33\) GHz\(.\mu m^6\) which is close to the VdW shift due to
the \(z\) component of the field. Larger than \(10 \mu\)m, the pair
energy is lightly attractive. This may be caused coupling with higher
order due to complicated Stark effect created by \(x\) component of the
electric field.

\paragraph{Perpendicular to plan of B field and atomic
pair}\label{perpendicular-to-plan-of-b-field-and-atomic-pair}

    \begin{Verbatim}[commandchars=\\\{\}]
{\color{incolor}In [{\color{incolor}26}]:} \PY{n}{builtins}\PY{o}{.}\PY{n}{theta} \PY{o}{=} \PY{n}{pi}\PY{o}{/}\PY{l+m+mi}{2} \PY{c}{\PYZsh{} angle between magnetic field (quantization axis) and atom pair}
         \PY{n}{builtins}\PY{o}{.}\PY{n}{Ffield} \PY{o}{=} \PY{l+m+mf}{1.e\PYZhy{}2} \PY{c}{\PYZsh{} V/cm}
         \PY{n}{builtins}\PY{o}{.}\PY{n}{theta\PYZus{}F} \PY{o}{=} \PY{l+m+mf}{0.1}\PY{o}{*}\PY{n}{pi}\PY{o}{/}\PY{l+m+mi}{2} \PY{c}{\PYZsh{} angle between F field and magnetic field}
         \PY{n}{builtins}\PY{o}{.}\PY{n}{phi\PYZus{}F} \PY{o}{=} \PY{l+m+mf}{1.}\PY{o}{*}\PY{n}{pi}\PY{o}{/}\PY{l+m+mi}{2} \PY{c}{\PYZsh{} angle between F\PYZus{}field and magnetic field , atom pair plan}
         \PY{o}{\PYZpc{}}\PY{k}{run} \PY{o}{\PYZhy{}}\PY{n}{n} \PY{n}{cal\PYZus{}VdW}\PY{o}{.}\PY{n}{py}
\end{Verbatim}

    \begin{Verbatim}[commandchars=\\\{\}]
atom 60C and atom 60C
theta = 90.0 deg
B\_field = 10.0 G
F\_field = 0.01 V/cm
theta\_F = 9.0 deg
phi\_F = 90.0 deg
Matrix size: 482
    \end{Verbatim}

    \begin{Verbatim}[commandchars=\\\{\}]
C:\textbackslash{}Users\textbackslash{}r14\textbackslash{}Documents\textbackslash{}GitHub\textbackslash{}test\textbackslash{}python\textbackslash{}cal\_VdW.py:271: ComplexWarning: Casting complex values to real discards the imaginary part
  out\_egr[i] , out\_vector[i] = np.linalg.eigh(EI + 1e18*V\_VdW* coef/(elm**3) + coef\_F*(V\_Stark1 + V\_Stark2))
    \end{Verbatim}

    \begin{center}
    \adjustimage{max size={0.9\linewidth}{0.9\paperheight}}{VdW for circular Rydberg atoms_files/VdW for circular Rydberg atoms_27_2.png}
    \end{center}
    { \hspace*{\fill} \\}
    
    \begin{center}
    \adjustimage{max size={0.9\linewidth}{0.9\paperheight}}{VdW for circular Rydberg atoms_files/VdW for circular Rydberg atoms_27_3.png}
    \end{center}
    { \hspace*{\fill} \\}
    
    \begin{Verbatim}[commandchars=\\\{\}]
C6 = [ 32.00307111] GHz.um\^{}6
At R = 4.901318761924517 um, prop > 5\% are:
pair of atom 60.0, 58.0, 58.0 and atom 60.0, 58.0, 58.0, at index 173
pair of atom 60C and atom 60C, at index 181
    \end{Verbatim}

    The declination of the electric field has small effect on the
interaction as expected. However it modifies a little bit when the atoms
are furthur away as the Stark effect increases the detuning b/w levels.
\(C_6 = 33\) GHz\(.\mu m^6\)

\subsubsection{B parallel to pair of atoms, F
perpendicular}\label{b-parallel-to-pair-of-atoms-f-perpendicular}

    \section{\texorpdfstring{\(|60C61C\rangle\)}{\textbar{}60C61C\textbackslash{}rangle}}\label{c61crangle}

\subsection{No electric field, B field parallel pair of
atoms}\label{no-electric-field-b-field-parallel-pair-of-atoms}

    \begin{Verbatim}[commandchars=\\\{\}]
{\color{incolor}In [{\color{incolor}39}]:} \PY{c}{\PYZsh{} Define levels builtins to globalize the parameters}
         \PY{k+kn}{import} \PY{n+nn}{builtins}
         \PY{n}{builtins}\PY{o}{.}\PY{n}{n1} \PY{o}{=} \PY{l+m+mi}{60}
         \PY{n}{builtins}\PY{o}{.}\PY{n}{l1} \PY{o}{=} \PY{n}{n1}\PY{o}{\PYZhy{}}\PY{l+m+mi}{1}
         \PY{n}{builtins}\PY{o}{.}\PY{n}{m1} \PY{o}{=} \PY{n}{l1}
         
         \PY{n}{builtins}\PY{o}{.}\PY{n}{n2} \PY{o}{=} \PY{l+m+mi}{61}
         \PY{n}{builtins}\PY{o}{.}\PY{n}{l2} \PY{o}{=} \PY{n}{n2}\PY{o}{\PYZhy{}}\PY{l+m+mi}{1}     
         \PY{n}{builtins}\PY{o}{.}\PY{n}{m2} \PY{o}{=} \PY{n}{l2}
         
         \PY{n}{builtins}\PY{o}{.}\PY{n}{Bfield} \PY{o}{=} \PY{l+m+mf}{10.e\PYZhy{}4} \PY{c}{\PYZsh{} Magnetic field from experiment 1 Teslta = 10*4 Gauss}
         \PY{n}{builtins}\PY{o}{.}\PY{n}{theta} \PY{o}{=} \PY{l+m+mi}{0}\PY{o}{*}\PY{n}{pi}\PY{o}{/}\PY{l+m+mi}{2} \PY{c}{\PYZsh{} angle between magnetic field (quantization axis) and atom pair}
         
         \PY{c}{\PYZsh{} Setup criterion}
         \PY{n}{delta\PYZus{}n\PYZus{}max} \PY{o}{=} \PY{l+m+mi}{6}
         \PY{n}{l\PYZus{}max} \PY{o}{=} \PY{l+m+mi}{2}
         \PY{n}{Choice} \PY{o}{=} \PY{l+m+mf}{1e7}\PY{c}{\PYZsh{} cut off energy for 1st order term, in Hz}
         \PY{n}{builtins}\PY{o}{.}\PY{n}{R\PYZus{}test} \PY{o}{=} \PY{l+m+mf}{1e\PYZhy{}6}
         \PY{n}{Choice2} \PY{o}{=} \PY{l+m+mi}{10}\PY{o}{*} \PY{n}{Choice}
         
         \PY{n}{builtins}\PY{o}{.}\PY{n}{Choice\PYZus{}F} \PY{o}{=} \PY{l+m+mf}{1e\PYZhy{}1} \PY{c}{\PYZsh{} cut off for Stark shift}
         
         \PY{n}{builtins}\PY{o}{.}\PY{n}{Ffield} \PY{o}{=} \PY{l+m+mi}{0}\PY{o}{*}\PY{l+m+mf}{1e\PYZhy{}1} \PY{c}{\PYZsh{} V/cm}
         \PY{n}{builtins}\PY{o}{.}\PY{n}{theta\PYZus{}F} \PY{o}{=} \PY{l+m+mf}{0.001}\PY{o}{*}\PY{n}{pi}\PY{o}{/}\PY{l+m+mi}{2} \PY{c}{\PYZsh{} angle between F field and magnetic field}
         \PY{n}{builtins}\PY{o}{.}\PY{n}{phi\PYZus{}F} \PY{o}{=} \PY{l+m+mf}{0.001}\PY{o}{*}\PY{n}{pi}\PY{o}{/}\PY{l+m+mi}{2} \PY{c}{\PYZsh{} angle between F\PYZus{}field and magnetic field , atom pair plan}
         \PY{o}{\PYZpc{}}\PY{k}{run} \PY{o}{\PYZhy{}}\PY{n}{n} \PY{n}{cal\PYZus{}VdW}\PY{o}{.}\PY{n}{py}
\end{Verbatim}

    \begin{Verbatim}[commandchars=\\\{\}]
atom 60C and atom 61C
theta = 0.0 deg
B\_field = 10.0 G
F\_field = 0.0 V/cm
theta\_F = 0.09 deg
phi\_F = 0.09 deg
Matrix size: 847
    \end{Verbatim}

    \begin{Verbatim}[commandchars=\\\{\}]
C:\textbackslash{}Users\textbackslash{}r14\textbackslash{}Documents\textbackslash{}GitHub\textbackslash{}test\textbackslash{}python\textbackslash{}cal\_VdW.py:271: ComplexWarning: Casting complex values to real discards the imaginary part
  out\_egr[i] , out\_vector[i] = np.linalg.eigh(EI + 1e18*V\_VdW* coef/(elm**3) + coef\_F*(V\_Stark1 + V\_Stark2))
    \end{Verbatim}

    \begin{center}
    \adjustimage{max size={0.9\linewidth}{0.9\paperheight}}{VdW for circular Rydberg atoms_files/VdW for circular Rydberg atoms_30_2.png}
    \end{center}
    { \hspace*{\fill} \\}
    
    \begin{center}
    \adjustimage{max size={0.9\linewidth}{0.9\paperheight}}{VdW for circular Rydberg atoms_files/VdW for circular Rydberg atoms_30_3.png}
    \end{center}
    { \hspace*{\fill} \\}
    
    \begin{Verbatim}[commandchars=\\\{\}]
C3 = [ 6.42475142] GHz.um\^{}3
C3 = [-6.31526973] GHz.um\^{}3
At R = 1.5595191668205517 um, prop > 5\% are:
pair of atom 59C and atom 62C, at index 189
pair of atom 62C and atom 59C, at index 190
pair of atom 60C and atom 61C, at index 285
pair of atom 61C and atom 60C, at index 286
    \end{Verbatim}

    \(|60C61C\rangle\) and \(61C60C\rangle\) are resonantly coupled
\(\longrightarrow 1/R^3\) behaviour.

\$C\_3A =6.41, C\_3S =6.32 \$ GHz\(\mu m^3\)

\subsubsection{pair of atoms perpendicular to the B
field}\label{pair-of-atoms-perpendicular-to-the-b-field}

    \begin{Verbatim}[commandchars=\\\{\}]
{\color{incolor}In [{\color{incolor}40}]:} \PY{n}{builtins}\PY{o}{.}\PY{n}{theta} \PY{o}{=} \PY{n}{pi}\PY{o}{/}\PY{l+m+mi}{2} \PY{c}{\PYZsh{} angle between magnetic field (quantization axis) and atom pair}
         \PY{o}{\PYZpc{}}\PY{k}{run} \PY{o}{\PYZhy{}}\PY{n}{n} \PY{n}{cal\PYZus{}VdW}\PY{o}{.}\PY{n}{py}
\end{Verbatim}

    \begin{Verbatim}[commandchars=\\\{\}]
atom 60C and atom 61C
theta = 90.0 deg
B\_field = 10.0 G
F\_field = 0.0 V/cm
theta\_F = 0.09 deg
phi\_F = 0.09 deg
Matrix size: 847
    \end{Verbatim}

    \begin{Verbatim}[commandchars=\\\{\}]
C:\textbackslash{}Users\textbackslash{}r14\textbackslash{}Documents\textbackslash{}GitHub\textbackslash{}test\textbackslash{}python\textbackslash{}cal\_VdW.py:271: ComplexWarning: Casting complex values to real discards the imaginary part
  out\_egr[i] , out\_vector[i] = np.linalg.eigh(EI + 1e18*V\_VdW* coef/(elm**3) + coef\_F*(V\_Stark1 + V\_Stark2))
    \end{Verbatim}

    \begin{center}
    \adjustimage{max size={0.9\linewidth}{0.9\paperheight}}{VdW for circular Rydberg atoms_files/VdW for circular Rydberg atoms_32_2.png}
    \end{center}
    { \hspace*{\fill} \\}
    
    \begin{center}
    \adjustimage{max size={0.9\linewidth}{0.9\paperheight}}{VdW for circular Rydberg atoms_files/VdW for circular Rydberg atoms_32_3.png}
    \end{center}
    { \hspace*{\fill} \\}
    
    \begin{Verbatim}[commandchars=\\\{\}]
C3 = [ 3.27819811] GHz.um\^{}3
C3 = [-3.09313414] GHz.um\^{}3
At R = 4.348644533681211 um, prop > 5\% are:
pair of atom 60.0, 58.0, 58.0 and atom 61.0, 59.0, 59.0, at index 259
pair of atom 61.0, 59.0, 59.0 and atom 60.0, 58.0, 58.0, at index 269
pair of atom 60C and atom 61C, at index 285
pair of atom 61C and atom 60C, at index 286
    \end{Verbatim}

    \begin{Verbatim}[commandchars=\\\{\}]
{\color{incolor}In [{\color{incolor}36}]:} \PY{k+kn}{from} \PY{n+nn}{IPython.display} \PY{k+kn}{import} \PY{n}{HTML}
         \PY{n}{HTML}\PY{p}{(}\PY{l+s}{\PYZsq{}\PYZsq{}\PYZsq{}}\PY{l+s}{\PYZlt{}script type=}\PY{l+s}{\PYZdq{}}\PY{l+s}{text/javascript}\PY{l+s}{\PYZdq{}}\PY{l+s}{\PYZgt{}}
         \PY{l+s}{    on = }\PY{l+s}{\PYZdq{}}\PY{l+s}{View input}\PY{l+s}{\PYZdq{}}\PY{l+s}{;}
         \PY{l+s}{    off = }\PY{l+s}{\PYZdq{}}\PY{l+s}{Hide input}\PY{l+s}{\PYZdq{}}
         \PY{l+s}{    function onoff()\PYZob{}}
         \PY{l+s}{      currentvalue = document.getElementById(}\PY{l+s}{\PYZsq{}}\PY{l+s}{onoff}\PY{l+s}{\PYZsq{}}\PY{l+s}{).value;}
         \PY{l+s}{      if(currentvalue == off)\PYZob{}}
         \PY{l+s}{        document.getElementById(}\PY{l+s}{\PYZdq{}}\PY{l+s}{onoff}\PY{l+s}{\PYZdq{}}\PY{l+s}{).value=on;}
         \PY{l+s}{          \PYZdl{}(}\PY{l+s}{\PYZsq{}}\PY{l+s}{div.input}\PY{l+s}{\PYZsq{}}\PY{l+s}{).hide();}
         \PY{l+s}{      \PYZcb{}else\PYZob{}}
         \PY{l+s}{        document.getElementById(}\PY{l+s}{\PYZdq{}}\PY{l+s}{onoff}\PY{l+s}{\PYZdq{}}\PY{l+s}{).value=off;}
         \PY{l+s}{          \PYZdl{}(}\PY{l+s}{\PYZsq{}}\PY{l+s}{div.input}\PY{l+s}{\PYZsq{}}\PY{l+s}{).show();}
         \PY{l+s}{      \PYZcb{}}
         \PY{l+s}{\PYZcb{}}
         \PY{l+s}{\PYZlt{}/script\PYZgt{}}
         \PY{l+s}{\PYZlt{}input type=}\PY{l+s}{\PYZdq{}}\PY{l+s}{button}\PY{l+s}{\PYZdq{}}\PY{l+s}{ class=}\PY{l+s}{\PYZdq{}}\PY{l+s}{ui\PYZhy{}button ui\PYZhy{}widget ui\PYZhy{}state\PYZhy{}default ui\PYZhy{}corner\PYZhy{}all ui\PYZhy{}button\PYZhy{}text\PYZhy{}only}\PY{l+s}{\PYZdq{}}\PY{l+s}{ value=}\PY{l+s}{\PYZdq{}}\PY{l+s}{Hide input}\PY{l+s}{\PYZdq{}}\PY{l+s}{ id=}\PY{l+s}{\PYZdq{}}\PY{l+s}{onoff}\PY{l+s}{\PYZdq{}}\PY{l+s}{ onclick=}\PY{l+s}{\PYZdq{}}\PY{l+s}{onoff();}\PY{l+s}{\PYZdq{}}\PY{l+s}{\PYZgt{}.}\PY{l+s}{\PYZsq{}\PYZsq{}\PYZsq{}}\PY{p}{)}
\end{Verbatim}

            \begin{Verbatim}[commandchars=\\\{\}]
{\color{outcolor}Out[{\color{outcolor}36}]:} <IPython.core.display.HTML at 0xa5ce128>
\end{Verbatim}
        
    \begin{Verbatim}[commandchars=\\\{\}]
{\color{incolor}In [{\color{incolor}}]:} 
\end{Verbatim}


    % Add a bibliography block to the postdoc
    
    
    
    \end{document}
